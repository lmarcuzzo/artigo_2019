
\section{Introduction}\label{INTRO}

Computer networks typically rely on dedicated equipment (\textit{i.e.}, middleboxes) to perform common network functions (\textit{e.g.}, NAT translation, intrusion detection and load balancing). However, despite the usual benefits of such middleboxes, including performance and reliability, they offer limited flexibility regarding the design and deployment of advanced network services. Network Functions Virtualization (NFV), in turn, is a networking paradigm that leverages virtualization technologies to decouple network functions from the physical equipment and run such functions as software that executes on commodity hardware~\cite{ETSI-2012}.

The adoption of NFV presents several benefits, including higher customization and the reduction of Capital and Operational Expenditures (CAPEX/OPEX). Multiple efforts are being conducted to foster the adoption of NFV technologies with the ultimate goal of encouraging the development of solid foundations that support advanced NFV solutions.

%New standards, models, and NFV enablers (systems used to support the execution of Virtual Network Functions - VNFs) are being developed by diverse organizations, such as the European Telecommunications Standards Institute (ETSI) and the Internet Engineering Task Force (IETF), as well as several working groups (\textit{e.g.}, ETSI NFV Industry Specification Groups - ISG, IRTF NFV Research Group - NFVRG, and IETF Service Function Chaining - SFC). The ultimate goal of these efforts is to encourage the development of solid foundations that support advanced NFV solutions.

Essentially, a VNF is divided into two main parts: the Network Function (NF) itself and the VNF platform.  NF corresponds to the software implementation responsible for packet processing, while the VNF platform is the environment that supports the execution of NFs. VNF platforms are designed taking into account the need to enable the creation of multiple network functions while consuming few computing resources. However, existing VNF platforms (\textit{e.g.}, ClickOS \cite{Martins-2014} and OpenNetVM \cite{Zhang-2016}) are not created using standardized architectures, thus resulting in solutions that are either proprietary or present serious limitations, such as the lack of support for advanced NFV specifications, in particular Network Service Header (NSH) -- a packet header that enables the creation of dynamic service planes \cite{Quinn-2018}.

%in particular VNF Components (VNFCs) \cite{ISG-2013} -- individual elements consisting of some or all of the VNF functionality -- and Network Service Header (NSH) -- a packet header that enables the creation of dynamic service planes \cite{Quinn-2018}.

%In this paper, we introduce a comprehensive architecture for VNF platforms that strictly adheres to ETSI requirements and provides support for VNFCs and NSH. Our key contributions are: i) the design of the core elements of a flexible architecture for VNF platforms that support VNFCs and NSH; ii) the development of a VNF platform prototype based on the proposed reference architecture; and iii) the identification of critical features provided by both VNFCs and NSH that enable the development of advanced network services.

In this paper, we introduce a comprehensive architecture for VNF platforms that strictly adheres to ETSI requirements and provides support for NSH. Our key contributions are: i) the development of a VNF platform prototype based on ETSI requirements; and ii) the identification of critical features provided by NSH that enables the development of advanced network services.

The remaining of this paper is organized as follows. Section II presents the background on NFV along with its main components, basic requirements, and a review of the literature. In Section III, we propose a architecture for developing VNFs with support for NSH, and instantiate this architecture by describing a running prototype platform implemented to support virtualized network functions. In Section IV, we evaluate the performance of our prototype. Finally, in Section V, we conclude this paper with final remarks and an outline of future work.
